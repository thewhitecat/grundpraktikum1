\documentclass[12pt,a4paper]{article}
\usepackage[utf8]{inputenc}
\usepackage[german]{babel}
\usepackage[T1]{fontenc}
\usepackage{amsmath}
\usepackage{amsfonts}
\usepackage{amssymb}
\usepackage{graphicx}
\usepackage{float}
\usepackage[left=2cm,right=2cm,top=2cm,bottom=2cm]{geometry}
\author{Tim}

\begin{document}
\section{Kondensator}
\subsection{Versuchsbeschreibung}
In diesem Teilversuch soll die Kapazität eines Kondensators durch Auf- und Entladung von diesem bestimmt werden.
\subsubsection{Aufladung}
Wird eine Spannung an einen Kondensator angelegt, so wird dieser geladen, bis die Quellspannung kompensiert wird. Der Ladevorgang dauert umso länger, je größer der Widerstand des Stromkreises ist.\\
Aus der Maschenregel folgt:
\begin{equation}
U_0 = U_R + U_C \Rightarrow U_0 - U_C = R\cdot C\cdot \dfrac{dU_c}{dt}
\label{Kondensator_DGL}
\end{equation}
Bei der Aufladung kann man daraus den Strom und die Spannung bestimmen:
\begin{equation}
U_C(t) = U_0 \cdot (1-e^{-\dfrac{t}{R\cdot C}})
\end{equation}
\begin{equation}
I(t) = I_0 \cdot e^{-\dfrac{t}{R\cdot C}}
\end{equation}
Dabei ist $\tau = R \cdot C$ die Zeitkonstante des R-C-Kreises.
\subsubsection{Entladung}
Bei der Entladung liegt keine externe Spannung mehr an. Es gilt also: $U_0 = 0$\\
Mit dieser Annahme und Gl. \ref{Kondensator_DGL} folgt:
\begin{equation}
U_C(t) = U_0 \cdot e^{-\dfrac{t}{R\cdot C}}
\end{equation}
\begin{equation}
I(t) = -I_0 \cdot e^{-\dfrac{t}{R\cdot C}}
\end{equation}
\subsection{Aufbau und Durchführung}

\subsection{Auswertung}
\end{document}