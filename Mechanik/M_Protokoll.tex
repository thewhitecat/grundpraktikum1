\documentclass[12pt,a4paper]{article}
\usepackage[utf8]{inputenc}
\usepackage[german]{babel}
\usepackage[T1]{fontenc}
\usepackage{amsmath}
\usepackage{amsfonts}
\usepackage{amssymb}
\usepackage{graphicx}
\usepackage{float}
\usepackage[left=2cm,right=2cm,top=2cm,bottom=2cm]{geometry}
\usepackage{siunitx}
\author{Moritz}

\begin{document}
\setlength{\parindent}{0pt} 
\begin{center}
{\LARGE Versuchsprotokoll}\\
\begin{large}
zum Grundpraktikum Physik Teil I\\[0.4cm]
an der RWTH Aachen\\
I. Physikalisches Institut B\\[4.5cm]
\Large\textbf{\textsl{Mechanik}}\\[4cm]
\normalsize\textit{vorgelegt\\von}\\[0.4cm]
\large{Moritz Berger\\Tim Herbermann\\Gerald Kolter\\Sebastian Siebert}\\[1cm]
\large \textit{Gruppe B07} \\ [3cm]
\large \textbf{Wintersemester 2016/2017}
\end{large}
\end{center}
\newpage

\tableofcontents
\newpage

\part{Trägheitsmomente}

\section{Grundlagen}
In diesem Versuch soll das Trägheitsmoment verschiedener Körper analysiert werden.\\
Dieses ist allgemein definiert über
\begin{equation}
J = \int r^2 dm.
\end{equation}
Wir werden das Trägheitsmoment aus der Schwingungsdauer einer Drillachse, die durch eine Spiralfeder zum schwingen gebracht wird, bestimmen. Dabei wird ein Zusammenhang zwischen Schwingungsperiode T und dem Trägheitsmoment J  aus der Definition des Drehmomentes M hergeleitet:
\begin{equation}
M = J\cdot \dot{\omega}\Rightarrow -D\cdot \phi = J\cdot \ddot{\phi}\Rightarrow \ddot{\phi} + \dfrac{D}{J}\cdot \phi = 0
\end{equation}
Die Lösung dieser Differentialgleichung ist eine harmonische Schwingung mit der Kreisfrequenz $\omega = \sqrt{\dfrac{D}{J}}$. Daraus folgt:
\begin{equation}
 T = 2\cdot \pi \sqrt{\dfrac{J}{D}}
\end{equation}
oder, falls man das Trägheitsmoment angeben möchte:
\begin{equation}
J = \dfrac{D}{4\cdot \pi^2}\cdot T^2
\end{equation}\\
\\
Der Versuch wird in drei Teile aufgeteilt.\\
Im ersten Teil soll das Trägheitsmoment von Massenpunkten in Abhängigkeit von dem Abstand zur Drehachse untersucht werden. Dabei wird als erstes das Direktionsmoment der Feder bestimmt.\\
Im zweiten Teil wird das Trägheitsmoment eines Hohlzylinders, eines Vollzylinders, einer Kugel und einer Kreisscheibe miteinander verglichen.\\
Im dritten Teil soll der Steinersche Satz
\begin{equation}
J = J_0+m \cdot R^2
\end{equation}
bestätigt werden.

\section{Aufbau und Durchführung}
\begin{figure}
\includegraphics[width=\linewidth]{Bilder/Aufbau.PNG}
\caption{Versuchsaufbau für die Bestimmung von Trägheitsmomenten}
\label{fig:Aufbau}
\end{figure}
Der grundsätzliche Aufbau aller drei oben erwähnten Teilversuche ist gleich. Er wird in Abbildung \ref{fig:Aufbau} dargestellt. Es wird ein kugelgelagterter Stab mit einem Stativ und mit einer Spiralfeder, die eine Schwingung erzeugen soll, befestigt. Auf den Stab können je nach Teilversuch verschiedene Aufsätze aufgesetzt werden, deren Trägheitsmoment gemessen wird. Am unteren Stabende werden an der Innenseite einer U-förmigen Gabel zwei Magnete so befestigt, dass sich Nord- und Südpol gegenüber liegen. Zwischen die Magnete wird eine Hallsonde geführt, die als Winkelaufnehmer dient. Sie wird mit der Spannungsquelle des CASSYs verbunden. Die entstehende Hall-Spannung wird ebenfalls mit dem CASSY gemessen. Vor dem Experimentieren wird die Hallsonde durch horizontales Drehen so justiert, dass die gemessene Spannung möglichst bei \SI{0}{V} liegt. \\
Alle Experimente wurden mit zwei Aufbauten durchgeführt. Dabei hatte ein Aufbau die Feder mit der Nummer 2 und der andere Aufbau Feder Nummer 3. Entsprechend wird im Folgenden auf die Aufbauten bzw. Federn verwiesen.


\section{Bestimmung des Direktionsmomentes}

\section{Bestimmung des Direktionsmomentes}

\begin{figure}
\begin{center}
\includegraphics[scale=0.5]{Bilder/Feder2Kerbe1}
\includegraphics[scale=0.5]{Bilder/Feder2Kerbe3}
\end{center}
\caption{Exemplarische Schwingungsverläufe für Gewichte auf der ersten und dritten Kerbe für Feder 2.}
\label{fig:RohdatenFeder2}
\end{figure}

\begin{figure}
\begin{center}
\includegraphics[scale=0.5]{Bilder/Feder3Kerbe1}
\includegraphics[scale=0.5]{Bilder/Feder3Kerbe3}
\end{center}
\caption{Exemplarische Schwingungsverläufe für Massenpunkte auf der ersten und dritten Kerbe für Feder 3.}
\label{fig:RohdatenFeder3}
\end{figure}


Einige exemplarische Schwingungsverläufe finden sich in Abbildung [hier Rohdaten Feder 2] für Feder 2 und in Abbildung \ref{fig:RohdatenFeder3} für Feder 3.\\

Vor Aufzeichnung der Schwingungsvorgänge wurden die beiden Kugeln gewogen sowie der Kerbenabstand bestimmt. Die ausgewiesenen statistischen Fehler ergeben sich aus der Ablesegenauigkeit, der systematische Fehler der Waage wurde auf \SI{0.1}{g} abgeschätzt. Der systematische Fehler des Maßbands ist durch die Güteklasse II gegeben. Die Ergebnisse beider Federn finden sich in Tabelle \ref{tab:Vormessungen}.\\


\begin{table}
\caption{Vormessungen zur Bestimmung des Direktionsmoments der Federn.}
\begin{center}
\begin{tabular}{|c|c|c|c|c|}
\hline
Feder & $m_1$[g] & $m_2$[g] & Kerbenabstand $d$[cm]\\
\hline
2 & 237.8 $\pm$ 0.03 $\pm$ 0.1 & 238.2 $\pm$ 0.03 $\pm$ 0.1 & 5.00 $\pm$ 0.03 $\pm$ 0.07 \\
\hline
3 & 238.50 $\pm$ 0.03 $\pm$ 0.1 & 238.40 $\pm$ 0.03 $\pm$ 0.1 & 4.99 $\pm$ 0.03 $\pm$ 0.07\\ 
\hline
\end{tabular}
\end{center}
\label{tab:Vormessungen}
\end{table}

hier blabla über die pestimmung de rperiode\\

Trägt man nun das Quadrat der Periodendauer gegen das Quadrat des Abstands der Massen zur Drehachse kann aus der sich ergebenden Steigung das Direktionsmoment D berechnet werden (vgl. Grundlagen).\\
Bei entsprechender Quadrierung der Messwerte ergibt sich mittels Fehlerfortpflanzung für die Periodendauer und den Abstand der n-ten Kerbe:

\begin{equation}
\sigma_{T^2}=2\cdot T \sigma_T \quad \quad \quad \sigma_{r_n^2}=2\cdot r_n \sigma_{r_n}
\end{equation}

Der lineare Zusammenhang ist dabei schon gut zu erkennen. Die Ergebnisse der linearen Regressionen finden sich in Abbildung \ref{fig:RegressionenD}

\begin{figure}
\begin{center}
\includegraphics[scale=0.5]{Bilder/Feder2RegD}
\includegraphics[scale=0.5]{Bilder/Feder3RegD}
\end{center}
\caption{Ergebnisse der Anpassung für Feder 2 (links) und Feder 3 (rechts)}
\label{fig:RegressionenD}
\end{figure}





Mit der Steigung a folgt für das Direktionsmoment
\begin{equation}
D = 4 \pi^2 \frac{m_w}{a} \quad \quad \quad
\sigma_D= D \sqrt{(\frac{\sigma_{m_2}}{m_w})^2+(\frac{\sigma_a}{a})^2}
\end{equation}

wobei $m_w=m_1+m_2$ ist. Die systematischen Fehler der beiden Einzelmassen sind dabei nicht unabhängig und werden linear addiert.\\
Die Ergebnisse der Direktionsmomente finden sich in Tabelle \ref{tab:Direktionsmomente}.

\begin{table}
\caption{Ergebnisse der Direktionsmomentbestimmung}
\begin{center}
\begin{tabular}{|c|c|c|c|}
\hline
Feder & $m_w$[g] & a[$\frac{s^2}{m^2}$] & D[mNm]\\
\hline
2 & 476.00 $\pm$ 0.04 $\pm$ 0.2 & $632.43 \pm 1.46$ & $29.71 \pm 0.07 \pm 0.01$ \\
\hline
3 & 476.90 $\pm$ 0.04 $\pm$ 0.2 & - &  - \\
\hline
\end{tabular}
\end{center}
\label{tab:Direktionsmomente}
\end{table}

Aus dem soeben bestimmten Direktionsmoment und dem y-Achsenabschnitt der Anpassung kann auch das Trägheitsmoment des Stabs ohne angebrachte Massen bestimmt werden und mit dem theoretisch erwarteten Wert verglichen werden.\\

\begin{equation}
J_{Stab}^{exp}=\frac{bD}{4\pi^2} \quad \quad \quad
\sigma_{J_{Stab}^{exp}}=J_{Stab}^{exp} \sqrt{(\frac{\sigma_b}{b})^2+(\frac{\sigma_D}{D})^2}
\end{equation}

Für den Wert aus der Theorie erwartet man

\begin{equation}
J_{Stab}^{theo}=\frac{1}{12} m L^2 \quad \quad \quad
\sigma_{J_{Stab}^{theo}}=\sqrt{(\frac{L^2 \sigma_m}{12})^2+(\frac{2Lm\sigma_L}{12})^2}
\end{equation}

wobei m und L Masse bzw. Länge des Stabs sind.\\
Eine Zusammenstellung der sich so ergebenden Trägheitsmomente findet sich in Tabelle


\begin{table}
\begin{center}
\begin{tabular}{|c|c|c|c|c|}
\hline
Feder & $m_{Stab}$[g] & $L$[cm] & $J_{Stab}^{theo}$[$g m^2$] & $J_{Stab}^{exp}$[$g m^2$]\\
\hline
2 & 131.6 $\pm$ 0.03 $\pm$ 0.1 & 60.9 $\pm$ 0.03 $\pm$ 0.07 & $4.067 \pm 0.004 \pm 0.01 $ & $ 4.165 \pm 0.026 \pm 0.001 $ \\
\hline
3 & 130.5 $\pm$ 0.03 $\pm$ 0.1 & 61.0 $\pm$ 0.03 $\pm$ 0.07 & $4.046 \pm 0.004 \pm  0.01 $ & - \\
\hline
\end{tabular}
\end{center}
\end{table}


\section{Vergleich von Trägheitsmomenten}

\subsection{Auswertung Zylinder}
Bei der Messung der Trägheitsmomente von Hohl- und Vollzylinder wird ein Aufnahmeteller auf die Drillachse gesteckt. Da dieser Teller ein eigenes Trägheitsmoment $J_T$ hat, muss dieses zusätzlich noch bestimmt werden. Dies geschieht über die Aufnahme von Schwingungs-Messungen mit nur dem Teller. Danach werden nacheinander entweder der Voll- oder Hohlzylinder auf den Teller gesteckt und ebenfalls in Schwingung versetzt. Das Trägheitsmoment des Hohlzylinders wird dann folgendermaßen bestimmt:
\begin{equation}
J_{HZ} = J_{gemessen}-J_T
\end{equation}
Analog gilt für den Vollzylinder:
\begin{equation}
J_{VZ} = J_{gemessen}-J_T
\end{equation}
\subsubsection{Rohdaten}
\begin{figure}
\includegraphics[width=0.49\linewidth]{Bilder/Teller_Rohdaten.PNG}
\includegraphics[width=0.49\linewidth]{Bilder/Teller_FFT.PNG}
\caption{Rohdaten der Tellermessung mit FFT zur Veranschaulichung.}
\end{figure}
\begin{figure}
\includegraphics[width=0.49\linewidth]{Bilder/Voll_Rohdaten.PNG}
\includegraphics[width=0.49\linewidth]{Bilder/Voll_FFT.PNG}
\includegraphics[width=0.49\linewidth]{Bilder/Hohl_Rohdaten.PNG}
\includegraphics[width=0.49\linewidth]{Bilder/Hohl_FFT.PNG}
\caption{Rohdaten der Zylinder(oben: Vollzylinder, unten: Hohlzylinder) jeweils mit einer FFT zur Veranschaulichung.}
\label{fig:Zylinder_Rohdaten}
\end{figure}

\section{Bestätigung des Steinerschen Satzes}

\end{document}