\documentclass[12pt,a4paper]{article}
\usepackage[utf8]{inputenc}
\usepackage[german]{babel}
\usepackage[T1]{fontenc}
\usepackage{amsmath}
\usepackage{amsfonts}
\usepackage{amssymb}
\usepackage{graphicx}
\usepackage{siunitx}
\usepackage[left=2cm,right=2cm,top=2cm,bottom=2cm]{geometry}
\author{Tim}

\begin{document}
	\setlength{\parindent}{0pt} 
	\begin{center}
		{\LARGE Versuchsprotokoll}\\
		\begin{large}
			zum Grundpraktikum Physik Teil II\\[0.4cm]
			an der RWTH Aachen\\
			I. Physikalisches Institut B\\[4.5cm]
			\Large\textbf{\textsl{E-Lehre}}\\[4cm]
			\normalsize\textit{vorgelegt\\von}\\[0.4cm]
			\large{Moritz Berger\\Tim Herbermann\\Gerald Kolter\\Sebastian Siebert}\\[1cm]
			\large \textit{Gruppe A07} \\ [3cm]
			\large \textbf{Sommersemester 2017}
		\end{large}
	\end{center}
	\newpage
	
	\tableofcontents
	\newpage


\section{Grundlagen}

\subsection{Versuchsbeschreibung}
In den folgenden Versuchen wurde die Güte von Serien- und Parallelschwingkreisen auf verschiedene Arten bestimmt. Zusätzlich wurde in einem letzten Aufbau ein Hoch- bzw. Tiefpass untersucht.

\subsection{Physikalische Grundlagen}

\paragraph{Serienkreis}
Aus der Kirchhoffschen Maschenregel ergibt sich eine Differentialgleichung zur Beschreibung der erzwungenen Schwingung.

\begin{equation}
\frac{U}{L} = \frac{d^2 Q}{dt^2}+\frac{R}{L}\frac{dQ}{dt}+\frac{1}{LC} Q
\end{equation}



Dabei ergibt sich der maximale Strom im Kreis bei Resonanz für
\begin{equation}
\omega_0 = \frac{1}{\sqrt{LC}}
\end{equation} 

Die Güte des Schwingkreises ist definiert durch
\begin{equation}
Q_s = \frac{1}{R} \sqrt{\frac{L}{C}}
\end{equation}

und beschreibt unter anderem den Zusammenhang zwischen angelegter Spannung und Spannungsüberhöhung im Resonanzfall:

\begin{equation}
\frac{U_L(\omega_0)}{U(\omega_0} = Q_s 
\end{equation}



\paragraph{Parallelschwingkreis}
Aus der Knotenregel kann ebenfalls eine Differentialgleichung zur Beschreibung des Parallelschwingkreises hergeleitet werden.

Aus dieser kann ebenfalls die Resonanzfrequenz bestimmt werden.

\begin{equation}
\omega_0 = \sqrt{\frac{1-\frac{C}{L}\cdot R_L^2}{LC}}
\end{equation} 

Für große Widerstände im Kreis ergibt sich für die Güte der Schaltung näherungsweise der Zusammenhang:

\begin{equation}
Q_P = \frac{1}{R_L} \sqrt{\frac{L}{C}}
\end{equation}

Im Falle des Parallelschwingkreises kommt es bei Resonanz zu einer Stromüberhöhung. Auch hier ergibt sich der Zusammenhang:

\begin{equation}
\frac{(I_L)_0}{I_0} = Q_p
\end{equation}

\paragraph{Hoch -und Tiefpass}
Ein Hoch- bzw. Tiefpass 1. Ordnung besteht aus einer Serienschaltung von Widerstand und Kondensator. Durch Anlegen einer Wechselspannung werden, abhängig davon wo die Spannung abgegriffen wird, entweder große oder kleine Frequenzen blockiert bzw. durchgelassen.

Der Hochpass ergibt sich durch Spannungsabgriff am Widerstand. Die Übertragsfunktion ergibt sich dabei zu:

\begin{equation}
\frac{U_a}{U_e} = \frac{1}{\sqrt{(\frac{1}{\omega C R})^2+1}}
\end{equation}

Den Tiefpass erhält man dann entsprechend durch Abgriff am Kondensator. Die Übertragsfunktion lautet:

\begin{equation}
\frac{U_a}{U_e} = \frac{1}{\sqrt{1+(\omega C R)^2}}
\end{equation}

\section{Serienschwingkreis}

\subsection{Aufbau und Durchführung}

\begin{figure}
\caption{}
\label{fig:AufbauSerie}
\end{figure}

\section{Parallelschwingkreis}

\section{Hoch- und Tiefpass}





\end{document}