\documentclass[12pt,a4paper]{article}
\usepackage[utf8]{inputenc}
\usepackage[german]{babel}
\usepackage[T1]{fontenc}
\usepackage{amsmath}
\usepackage{amsfonts}
\usepackage{amssymb}
\usepackage{graphicx}
\usepackage[left=2cm,right=2cm,top=2cm,bottom=2cm]{geometry}
\author{Tim}

\begin{document}

\tableofcontents
\newpage

\section{Prismenspektrometer}

\subsection{Grundlagen}

\subsection{Dispersionskurve}
In diesem Teilversuch wurde die Dispersionskurve des verwendeten Prismas mithilfe einer Quecksilber-Cadmium-Lampe vermessen.

\subsection{Dispersionskurve}
In diesem Teilversuch wurde die Dispersionskurve des verwendeten Prismas mithilfe einer Quecksilber-Cadmium-Lampe vermessen.

\subsubsection{Aufbau und Durchführung}
Durch eine Kollimatorlinse werden ebene Wellenfronten erzeugt. Diese werden mit dem einem Drehteller befindlichen Prisma gebrochen. Das gebrochene Spaltbild wird mit einem schwenkbaren Fernrohr beobachtet. Die relative Position des Fernrohrs kann dabei auf 1' genau abgelesen werden.

In diesem Versuch werden zunächst die Minimalablenkungen des Prismas für mehrere Spektrallinien bestimmt.
Durch gleichsinniges Drehen von Prismenteller und Fernrohr wandern auch die Spektrallinien, zunächst in die eine und dann in die andere Richtung. Der Umkehrpunkt entspricht dabei der Minimalablenkung. Man wiederholt die Messung, diesmal trifft das Licht allerdings auf der anderen Prismenseite auf. Dann gilt für die Minimalablenkung

\begin{equation}
2 \delta_{min} = \psi_2-\psi_1
\end{equation}

wobei $\psi_i$ die Umkehrpunkte sind.
Zur Bestimmung der statistischen Unsicherheit auf die $\psi_i$ wurde eine Rauschmessung an der $\psi_1$-Position der blau-grünen Cadmiumlinie ($\lambda=508,58nm$) durchgeführt.

Für die übrigen Positionen der Spektrallinien wurde die Messung drei (bzw. fünf) mal wiederholt. Der Fehler auf die sich so ergebenden Mittelwert ist dann $\sigma_{\psi}=\frac{\sigma_{psi}}{\sqrt{N}}$.

Es ergeben sich damit die Minimalablenkungen und der Fehler auf selbige für jede vermessene Spektrallinie.

\begin{equation}
\sigma_{\delta} = \sqrt{\sigma_{\psi_1}^2+\sigma_{psi_2}^2}
\end{equation}

Damit folgt für die Brechungsindizes bei den entsprechenden Wellenlängen::

\begin{equation}
n = \frac{sin(\frac{\delta_{min}+\epsilon}{2})}{sin(\frac{\epsilon}{2})}
\end{equation}

\begin{equation}
\sigma_n = \frac{cos(\frac{\delta_{min}+\epsilon}{2})}{sin(\frac{\epsilon}{2})} \frac{\delta_{min}}{2}
\end{equation}

\subsubsection{Rohdaten}

\begin{table}
\begin{tabular}{|c|c|}
\hline 
$\psi_1$ & $\psi_2$ \\ 
\hline 
25$^{\circ}$3' & 147$^{\circ}$21' \\ 
\hline 
25$^{\circ}$1' & 147$^{\circ}$22' \\ 
\hline 
25$^{\circ}$0'& 147$^{\circ}$13' \\ 
\hline 
25$^{\circ}$1' & 147$^{\circ}$11' \\ 
\hline 
25$^{\circ}$1' & 147$^{\circ}$16' \\ 
\hline 
25$^{\circ}$3' & \\ 
\hline 
25$^{\circ}$9' & \\ 
\hline 
25$^{\circ}$3' & \\ 
\hline 
25$^{\circ}$10' & \\ 
\hline 
25$^{\circ}$20' & \\ 
\hline 
25$^{\circ}$14' & \\ 
\hline 
25$^{\circ}$11' & \\ 
\hline 
25$^{\circ}$9' & \\ 
\hline 
25$^{\circ}$8' & \\ 
\hline 
25$^{\circ}$10' & \\ 
\hline 
\end{tabular} 
\label{tab:RauschenPrisma}
\caption{Rohdaten der grün-blauen Cadmiumlinie. Die Messwerte auf $\psi_1$ dienen als Rauschmessung für die statistische Unsicherheit .}
\end{table}

\subsubsection{Auswertung}



\section{Gitterspektrometer}


\end{document}